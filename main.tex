\documentclass{article}
\usepackage[utf8]{inputenc}
\usepackage[a4paper, total={6.5in, 9.5in}]{geometry}
\usepackage[portuguese]{babel}
\usepackage[hidelinks]{hyperref}
\usepackage{amsmath}
\usepackage{amsthm}
\usepackage{amsfonts}
\usepackage{amssymb}
\usepackage{xcolor}
\usepackage{graphicx}
\usepackage{tikz}
\usetikzlibrary{babel,arrows,positioning,chains,matrix,scopes,cd,quotes,calc,decorations.pathmorphing}
\usepackage{caption}
\usepackage{subcaption}
\usepackage{algorithmic}
\usepackage[portuguese, ruled, lined]{algorithm2e}
\usepackage{setspace}
\usepackage[T1]{fontenc}
\usepackage{biblatex}
\addbibresource{bib.bib}
\usepackage{csquotes}
\usepackage{pythonhighlight}
\usepackage{algorithmic}
\usepackage[portuguese, ruled, lined]{algorithm2e}

%quiver
\tikzset{curve/.style={settings={#1},to path={(\tikztostart)
    .. controls ($(\tikztostart)!\pv{pos}!(\tikztotarget)!\pv{height}!270:(\tikztotarget)$)
    and ($(\tikztostart)!1-\pv{pos}!(\tikztotarget)!\pv{height}!270:(\tikztotarget)$)
    .. (\tikztotarget)\tikztonodes}},
    settings/.code={\tikzset{quiver/.cd,#1}
        \def\pv##1{\pgfkeysvalueof{/tikz/quiver/##1}}},
    quiver/.cd,pos/.initial=0.35,height/.initial=0}

% TikZ arrowhead/tail styles.
\tikzset{tail reversed/.code={\pgfsetarrowsstart{tikzcd to}}}
\tikzset{2tail/.code={\pgfsetarrowsstart{Implies[reversed]}}}
\tikzset{2tail reversed/.code={\pgfsetarrowsstart{Implies}}}
% TikZ arrow styles.
\tikzset{no body/.style={/tikz/dash pattern=on 0 off 1mm}}

\newtheorem{definition}{Definição}
\newtheorem{proposition}[definition]{Proposição}
\newtheorem{lemma}[definition]{Lema}
\newtheorem{axiom}[definition]{Axioma}
\newtheorem{corollary}[definition]{Corolário}
\newtheorem{theorem}[definition]{Teorema}
\newtheorem{distribution}[definition]{Distribuição}
\newtheorem{example}[definition]{Exemplo}

\newcommand{\red}[1]{{\color{red} #1}}
\newcommand{\blue}[1]{{\color{blue} #1}}
\DeclareMathOperator{\freeab}{Free_{Ab}}

\DeclareMathOperator{\Top}{Top}
\DeclareMathOperator{\Ab}{Ab}
\DeclareMathOperator{\Grp}{Grp}
\DeclareMathOperator{\im}{Im}
\DeclareMathOperator{\module}{Mod}
\DeclareMathOperator{\Int}{Int}
\DeclareMathOperator{\coker}{coker}
\DeclareMathOperator{\chain}{Ch}
\DeclareMathOperator{\Hom}{Hom}
\DeclareMathOperator{\sign}{sign}
\DeclareMathOperator*{\argmax}{arg\,max}
\DeclareMathOperator*{\argmin}{arg\,min}
\DeclareMathOperator{\Alt}{Alt}
\DeclareMathOperator{\sgn}{sgn}
\DeclareMathOperator{\supp}{supp}
\DeclareMathOperator{\cl}{cl}
\DeclareMathOperator{\End}{End}
\DeclareMathOperator{\grad}{grad}
\DeclareMathOperator{\Exp}{Exp}
\DeclareMathOperator{\codim}{codim}
\DeclareMathOperator{\Ricci}{Ricci}
\DeclareMathOperator{\arccosh}{arccosh}
\DeclareMathOperator{\gyr}{gyr}
\DeclareMathOperator{\arctanh}{arctanh}
\DeclareMathOperator{\Log}{Log}
\DeclareMathOperator{\ArithmeticMod}{mod}

\newcommand{\modulus}[1]{\phantom{.}(\ArithmeticMod\phantom{.}#1)}

\newcommand{\Mod}[1]{$\module_{#1}$}
\newcommand{\Chain}[1]{$\chain(#1)$}

\newcommand{\openset}[0]{{\phantom{}\subset}{\circ}\phantom{.}}

\def\arrvline{\hfil\kern\arraycolsep\vline\kern-\arraycolsep\hfilneg}

\begin{document}
\begin{titlepage}
    \center
    \textsc{\textbf{\Large Iniciação Científica}}\\[0.5cm]
    \textsc{\textbf{\large Relatório Final}}\\[0.5cm]
    
    \vspace{4cm}
    
    \rule{\linewidth}{0.5mm}\\[0.4cm]
    {\huge \textbf{Desigualdades de Clarkson}}
    \rule{\linewidth}{0.5mm}\\[1.0cm]
    
    \vspace{1.0cm}
    
    \textbf{\Large{Universidade de São Paulo}}\\[0.2cm]
    \textbf{\large{Instituto de Ciências Matemáticas e Computação}}
    \vspace{4cm}
    
    \begin{flushright}
        \begin{tabular}{@{}ll@{}}
            \hspace{1cm}\textbf{\large{Aluno:}} & \textbf{\large{Lucas Giraldi Almeida Coimbra}}\\
            \hspace{1cm}\textbf{\large{Orientador:}} & \textbf{\large{Éder Rítis Aragão Costa}}\\
            \end{tabular}
    \end{flushright}
    
    \vfill
    \Large{Agosto de 2024}\\[0.2cm]
    \Large{São Carlos}
\end{titlepage}

\newpage

\tableofcontents

\newpage

\section{Teoria da Medida e Espaços \texorpdfstring{$L^p$}{Lp}}

O primeiro passo para o estudo das desigualdades de Clarkson é a construção de alguns conceitos de teoria da medida.

\subsection{O problema da medida em \texorpdfstring{$\mathbb{R}^n$}{Rn}}

Considere a família $\mathcal{P}(\mathbb{R}^n)$ de todos os subconjuntos de $\mathbb{R}^n$. Se $\{a_i\}_{i = 1}^n$ e $\{b_i\}_{i = 1}^n$ são conjuntos em $\mathbb{R}$ tais que $a_i < b_i$ para todo $i = 1, \dots, n$, então podemos definir $$\ell\left(\prod_{i = 1}^n [a_i, b_i)\right) = \prod_{i = 1}^n |b_i - a_i|$$ como o \textit{volume} do $n$-retângulo dado pelo cartesiano acima. O que procuramos a seguir é uma função $\ell \colon \mathcal{P}(\mathbb{R}^n) \to [0, \infty]$ que estenda de maneira ``minimamente razoável'' o volume de $n$-retângulos em $\mathbb{R}^n$. Com ``minimamente razoável'', queremos dizer que a nossa esperança é de que: \begin{itemize}
    \item se $x \in \mathbb{R}^n$ e $A \subset \mathbb{R}^n$, então $$\ell(A + x) = \ell(A);$$
    \item se $\{A_i\}_{i = 1}^n \subset \mathcal{P}(\mathbb{R}^n)$ é tal que $A_i \cap A_j = \emptyset$ para $i \neq j$, então $$\ell\left(\bigcup_{i = 1}^\infty A_i\right) = \sum_{i = 1}^\infty \ell(A_i).$$ 
\end{itemize}
O texto produzido aqui pode ser de cunho acadêmico, mas o que seria de nós, autores, se não ensinássemos alguma coisa sobre a vida fora da matemática? Nossa primeira lição, portanto, é que esperança não serve de nada. Provaremos isso mostrando que tal extensão $\ell$ não pode existir. Para cumprir tal tarefa, definimos uma relação: se $x, y \in \mathbb{R}^n$ escrevemos $$x \sim y \quad \text{se, e somente se,} \quad x - y \in \mathbb{Q}^n.$$

\begin{proposition}
    A relação definida acima é de equivalência.
\end{proposition}
\begin{proof}
    De fato, \begin{itemize}
        \item a relação é simétrica, já que $x - x = 0 \in \mathbb{Q}^n$;
        \item a relação é reflexiva, já que se $x - y \in \mathbb{Q}^n$, então $- (x - y) \in \mathbb{Q}^n$, e $-(x-y) = y-x$;
        \item a relação é transitiva, já que se $x - y, y - z \in \mathbb{Q}^n$, então $x - z = (x - y) - (y - z) \in \mathbb{Q}^n$.
    \end{itemize}
\end{proof}

Dado $x \in \mathbb{R}^n$, denotamos sua classe de equivalência sob a relação definida acima por $[x]$. Definimos $\{x\} \in [0,1)^n$ como o vetor que, em cada coordenada, possui a parte fracionária da respectiva entrada de $x$. Como $x - \{x\} \in \mathbb{Z}^n$, então $\{x\} \sim x$. Dessa forma podemos, utilizando o axioma da escolha, construir um conjunto $I \subset [0,1)$ que possui precisamente um elemento para representar cada classe de equivalência de $\sim$. O próximo passo então é definir, para cada $x \in \mathbb{Q}^n$, um conjunto $$I_x = \{y + x \modulus{1} \mid y \in I\}$$ e supor que a função $\ell$ procurada de fato existe, para chegarmos em uma contradição.

\begin{proposition}
    Para cada $x \in (\mathbb{Q} \cap [0,1))^n$, vale que $\ell(I_x) = \ell(I)$.
\end{proposition}
\begin{proof}
    Para provarmos a proposição, iremos particionar $I_x$ em $2^n$ pedaços. Para cada $i = 0, \dots, 2^n - 1$ definimos $e_i$ como sendo o vetor que contém, da esquerda para a direita em cada coordenada, a representação em base $2$ do número $i$. Definimos então $V_i = [0,1)^n + e_i$. Note que $$[0,2)^n = \bigsqcup_{i = 1}^{2^n} V_i$$ e portanto, definindo $U_i = (I + x) \cap V_i$, temos $$I + x = \bigsqcup_{i = 1}^{2^n} U_i.$$ O objetivo agora é provar que $$I_x = \bigsqcup_{i = 1}^{2^n} U_i - e_i.$$ Primeiro, note que a união é disjunta, afinal, se $x \in (U_i - e_i) \cap (U_j - e_j)$ para $i \neq j$, então $x = v_i - e_i = v_j - e_j$ para $v_i \in U_i$ e $v_j \in U_j$. Dessa maneira, $(v_i - x) = (v_j - x) + (e_i - e_j)$ e portanto $v_i - x \sim v_j - x$. Acontece, porém, que $v_i, v_j \in I+x$, portanto $v_i - x, v_j - x \in I$ e, como estes são equivalentes sob a relação definida, obrigatoriamente são iguais, da onde segue que $v_i = v_j$ e assim $v_i - e_i = v_i - e_j$, assim $e_i = e_j$ e portanto $i = j$, o que é um absurdo.

    Agora, provamos a igualdade desejada. Se $z \in I_x$, então existe $y \in I$ tal que $z = y + x \modulus{1}$. Tal $y$ é único, afinal, se existissem $y_1$ e $y_2$ satisfazendo a igualdade, então teríamos $y_1 \sim y_2$ com ambos em $I$, o que é um absurdo. Sabemos ainda que existe $i = 0, \dots, 2^n - 1$ com $y - e_i \in [0,1)^n$, ou seja, $y - e_i$ é a redução de $y$ módulo $1$, ou seja, $y - e_i = z - x$. Note agora que $y + x \in U_i$, afinal, $y + x$ pertence a $I + x$ e $y + x - e_i = z \in I_x \subset [0,1)^n$, assim $y + x \in V_i$. Dessa forma, $z \in U_i - e_i$, assim $$I_x \subset \bigsqcup_{i = 1}^{2^n} U_i - e_i.$$

    Para a outra inclusão, se $z \in U_j - e_j$ para algum $j$, então $z \in [0,1)^n$, afinal, $z = u - e_j$ para algum $u \in U_j$ e, como $u \in V_j = [0,1)^n + e_j$, segue que $u - e_j = z \in [0,1)^n$. Mais ainda, $u \in U_j$, ou seja, $u \in I + x$ e assim $u = y + x$ para $y \in I$, assim $z = y + x \modulus{1}$, portanto $z \in I_x$, da onde segue que $$I_x \supset \bigsqcup_{i = 1}^{2^n} U_i - e_i.$$

    Com a igualdade provada, o resultado segue, afinal, $$\ell(I_x) = \ell\left(\bigsqcup_{i = 1}^{2^n} U_i - e_i\right) = \sum_{i = 1}^{2^n} \ell(U_i - e_i) = \sum_{i = 1}^{2^n} \ell(U_i) = \ell\left(\bigsqcup_{i = 1}^{2^n} U_i\right) = \ell(I + x) = \ell(I).$$
\end{proof}

Antes de prosseguirmos, precisamos clarificar uma questão de notação: $x = y \modulus{k}$ significa que $x$ é o resto da divisão de $y$ por $k$, enquanto $x \equiv y \modulus{k}$ significa que $x$ e $y$ possuem o mesmo resto quando divididos por $k$.

\begin{proposition}\label{prop3}
    Se $Q = \mathbb{Q}^n \cap [0,1)^n$, então vale a igualdade $$[0,1)^n = \bigsqcup_{x \in Q} I_x.$$
\end{proposition}
\begin{proof}
    Uma das inclusões é simples, afinal, $I_x \subset [0,1)^n$ para todo $x \in Q$. Para a outra inclusão, se $z \in [0,1)^n$, então tome $y$ como sendo o único representante de $[z]$ em $I$. Sabemos que $z = y + x$ para algum $\mathbb{Q}^n$. Mais do que isso, se $k = x \modulus{1}$, então $k \in Q$ e $z = y + k \modulus{1}$, portanto $z \in I_k$ e assim $$[0,1)^n = \bigcup_{x \in Q} I_x.$$
    
    Resta agora provarmos que essa união é disjunta. Se $z \in I_x \cap I_y$ com $x \neq y$, então $z = i_1 + p$ e $z = i_2 + q$ com $i_1, i_2 \in I$, $p, q \in \mathbb{Q}^n$, $x = p \modulus{1}$ e $y = q \modulus{1}$. Dessa forma, $i_1 - i_2 = q - p \in \mathbb{Q}^n$, portanto $i_1 \sim i_2$ e assim $i_1 = i_2$ pois ambos estão em $I$, da onde segue que $q - p = 0$ e portanto $p = q$, assim, como $x = p \modulus{1}$ e $y = q \modulus{1}$, então $x = y$, o que é um absurdo.
\end{proof}

Com estas duas proposições, podemos concluir o absurdo. Pela Proposição \ref{prop3} temos $$\ell([0,1)^n) = \ell\left(\bigsqcup_{x \in Q} I_x\right) = \sum_{x \in Q} \ell(I_x) = \sum_{x \in Q} \ell(I)$$ e, como $\ell([0,1)^n) = 1$, a igualdade é um absurdo, afinal, se $\ell(I) > 0$ então a soma explode ao infinito, e se $\ell(I) = 0$ então a soma é nula. Todo esse absurdo foi gerado por uma simples suposição: existe esse mapa $\ell \colon \mathcal{P}(\mathbb{R}^n) \to [0, \infty]$. Provada a sua não-existência, a pergunta que nos resta é: será que é possível estender a noção de volume para uma classe menor de subconjuntos?

\subsection{Álgebras, medidas e Lebesgue}

Dado um conjunto $\Omega$, uma \textit{$\sigma$-álgebra} em $\Omega$ é uma família $\mathcal{F}$ que satisfaz algumas condições: \begin{itemize}
    \item $\Omega \in \mathcal{F}$;
    \item se $A \in \mathcal{F}$, então $\Omega \setminus A \in \mathcal{F}$;
    \item se $\{A_i\}_{i = 1}^\infty \subset \mathcal{F}$, então $$\bigcup_{i = 1}^\infty A_i \in \mathcal{F}.$$
\end{itemize}

Teoria da medida compartilha muito de suas bases com a teoria de probabilidade (que se resume, em certo sentido, à teoria de medidas finitas), e dessa forma podemos pensar em $\sigma$-álgebras como objetos extremamente probabilísticos. Essa família de subconjuntos de $\Omega$ representa a família de todos os eventos dos quais queremos calcular a probabilidade de acontecimento. Com essa visão, as propriedades se traduzem para: \begin{itemize}
    \item o evento ``qualquer coisa acontece'' existe e podemos calcular sua probabilidade;
    \item se podemos calcular a probabilidade de um evento $A$ acontecer, então podemos calcular a probabilidade de $A$ não acontecer, ou seja, a probabilidade de $\Omega \setminus A$ acontecer;
    \item se é possível calcular a probabilidade de cada $A_i$ acontecer para cada $i \in \mathbb{Z}_{>0}$, então podemos calcular a probabilidade de acontecer ao menos um dentre todos esses eventos, ou seja, a probabilidade de $$\bigcup_{i = 1}^\infty A_i$$ acontecer.
\end{itemize}

Elementos de uma $\sigma$-álgebra $\mathcal{F}$ são chamados de \textit{conjuntos $\mathcal{F}$-mensuráveis} ou apenas \textit{$\mathcal{F}$-mensuráveis}. O nome da $\sigma$-álgebra pode ser e será omitido quando esta for clara a partir do contexto. O próximo passo agora é arranjar uma maneira de calcular probabilidades. Se $\mathcal{F}$ é uma $\sigma$-álgebra em $\Omega$, dizemos que o par $(\Omega, \mathcal{F})$ é um \textit{espaço mensurável}. Uma \textit{medida} em $(\Omega, \mathcal{F})$ é um mapa $\mu \colon \mathcal{F} \to [0, \infty]$ que é $\sigma$-aditivo, ou seja, tal que $$\mu\left(\bigsqcup_{i = 1}^\infty A_i\right) = \sum_{i = 1}^\infty \mu(A_i)$$ para qualquer coleção $\{A_i\}_{i = 1}^\infty$ de mensuráveis disjuntos dois a dois. A tripla $(\Omega, \mathcal{F}, \mu)$ é dita um \textit{espaço de medida}. Temos ainda os casos especiais: \begin{itemize}
    \item se $\mu(\Omega) < \infty$, dizemos que $\mu$ é \textit{finita} e $(\Omega, \mathcal{F}, \mu)$ é um \textit{espaço de medida finita};
    \item se $\mu(\Omega) = 1$, dizemos que $\mu$ é uma \textit{medida de probabilidade} e $(\Omega, \mathcal{F}, \mu)$ é um \textit{espaço de probabilidade}. Nesse caso, costumamos trocar a notação $\mu$ por $\mathbb{P}$;
    \item se existe uma família $\{A_i\}_{i = 1}^\infty$ de mensuráveis tais que $\mu(A_i) < \infty$ para todo $i$ e $$\Omega = \bigcup_{i = 1}^\infty A_i,$$ dizemos que $\mu$ é \textit{$\sigma$-finita} e $(\Omega, \mathcal{F}, \mu)$ é um \textit{espaço de medida $\sigma$-finita}.
\end{itemize}

Existem diversos exemplos de espaços mensuráveis e de medida. Esse texto tem o objetivo de fazer as construções necessárias apenas para ter os pré-requisitos durante o estudo das desigualdades de Clarkson. Dessa maneira, vamos ignorar alguns exemplos que são costumeiramente feitos e focar logo nos dois mais importantes.

Se $S \subset \mathcal{P}(\Omega)$ então, como $\mathcal{P}(\Omega)$ é claramente uma $\sigma$-álgebra, a família $\mathcal{A}$ de todas as $\sigma$-álgebras contendo $S$ é não vazia. Dessa maneira, podemos definir $$\sigma(S) = \bigcap_{A \in \mathcal{A}} A.$$ Esse conjunto é também uma $\sigma$-álgebra, chamada de \textit{$\sigma$-álgebra gerada por $S$}. Supondo agora que $\Omega$ é topológico e $\tau$ é o conjunto de seus abertos, chamamos $\sigma(\tau)$ de \textit{$\sigma$-álgebra de Borel} em $\Omega$.

\nocite{*}
\printbibliography
\end{document}