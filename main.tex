\documentclass{article}
\usepackage[utf8]{inputenc}
\usepackage[a4paper, total={6.5in, 9.5in}]{geometry}
\usepackage[portuguese]{babel}
\usepackage[hidelinks]{hyperref}
\usepackage{amsmath}
\usepackage{amsthm}
\usepackage{amsfonts}
\usepackage{amssymb}
\usepackage{xcolor}
\usepackage{graphicx}
\usepackage{tikz}
\usetikzlibrary{babel,arrows,positioning,chains,matrix,scopes,cd,quotes,calc,decorations.pathmorphing}
\usepackage{caption}
\usepackage{subcaption}
\usepackage{algorithmic}
\usepackage[portuguese, ruled, lined]{algorithm2e}
\usepackage{setspace}
\usepackage[T1]{fontenc}
\usepackage{biblatex}
\addbibresource{bib.bib}
\usepackage{csquotes}
\usepackage{pythonhighlight}
\usepackage{algorithmic}
\usepackage[portuguese, ruled, lined]{algorithm2e}

%quiver
\tikzset{curve/.style={settings={#1},to path={(\tikztostart)
    .. controls ($(\tikztostart)!\pv{pos}!(\tikztotarget)!\pv{height}!270:(\tikztotarget)$)
    and ($(\tikztostart)!1-\pv{pos}!(\tikztotarget)!\pv{height}!270:(\tikztotarget)$)
    .. (\tikztotarget)\tikztonodes}},
    settings/.code={\tikzset{quiver/.cd,#1}
        \def\pv##1{\pgfkeysvalueof{/tikz/quiver/##1}}},
    quiver/.cd,pos/.initial=0.35,height/.initial=0}

% TikZ arrowhead/tail styles.
\tikzset{tail reversed/.code={\pgfsetarrowsstart{tikzcd to}}}
\tikzset{2tail/.code={\pgfsetarrowsstart{Implies[reversed]}}}
\tikzset{2tail reversed/.code={\pgfsetarrowsstart{Implies}}}
% TikZ arrow styles.
\tikzset{no body/.style={/tikz/dash pattern=on 0 off 1mm}}

\newtheorem{definition}{Definição}
\newtheorem{proposition}[definition]{Proposição}
\newtheorem{lemma}[definition]{Lema}
\newtheorem{axiom}[definition]{Axioma}
\newtheorem{corollary}[definition]{Corolário}
\newtheorem{theorem}[definition]{Teorema}
\newtheorem{distribution}[definition]{Distribuição}
\newtheorem{example}[definition]{Exemplo}

\newcommand{\red}[1]{{\color{red} #1}}
\newcommand{\blue}[1]{{\color{blue} #1}}
\DeclareMathOperator{\freeab}{Free_{Ab}}

\DeclareMathOperator{\Top}{Top}
\DeclareMathOperator{\Ab}{Ab}
\DeclareMathOperator{\Grp}{Grp}
\DeclareMathOperator{\im}{Im}
\DeclareMathOperator{\module}{Mod}
\DeclareMathOperator{\Int}{Int}
\DeclareMathOperator{\coker}{coker}
\DeclareMathOperator{\chain}{Ch}
\DeclareMathOperator{\Hom}{Hom}
\DeclareMathOperator{\sign}{sign}
\DeclareMathOperator*{\argmax}{arg\,max}
\DeclareMathOperator*{\argmin}{arg\,min}
\DeclareMathOperator{\Alt}{Alt}
\DeclareMathOperator{\sgn}{sgn}
\DeclareMathOperator{\supp}{supp}
\DeclareMathOperator{\cl}{cl}
\DeclareMathOperator{\End}{End}
\DeclareMathOperator{\grad}{grad}
\DeclareMathOperator{\Exp}{Exp}
\DeclareMathOperator{\codim}{codim}
\DeclareMathOperator{\Ricci}{Ricci}
\DeclareMathOperator{\arccosh}{arccosh}
\DeclareMathOperator{\gyr}{gyr}
\DeclareMathOperator{\arctanh}{arctanh}
\DeclareMathOperator{\Log}{Log}

\newcommand{\Mod}[1]{$\module_{#1}$}
\newcommand{\Chain}[1]{$\chain(#1)$}

\newcommand{\openset}[0]{{\phantom{}\subset}{\circ}\phantom{.}}

\def\arrvline{\hfil\kern\arraycolsep\vline\kern-\arraycolsep\hfilneg}

\begin{document}
\begin{titlepage}
    \center
    \textsc{\textbf{\Large Iniciação Científica}}\\[0.5cm]
    \textsc{\textbf{\large Relatório Final}}\\[0.5cm]
    
    \vspace{4cm}
    
    \rule{\linewidth}{0.5mm}\\[0.4cm]
    {\huge \textbf{Desigualdades de Clarkson}}
    \rule{\linewidth}{0.5mm}\\[1.0cm]
    
    \vspace{1.0cm}
    
    \textbf{\Large{Universidade de São Paulo}}\\[0.2cm]
    \textbf{\large{Instituto de Ciências Matemáticas e Computação}}
    \vspace{4cm}
    
    \begin{flushright}
        \begin{tabular}{@{}ll@{}}
            \hspace{1cm}\textbf{\large{Aluno:}} & \textbf{\large{Lucas Giraldi Almeida Coimbra}}\\
            \hspace{1cm}\textbf{\large{Orientador:}} & \textbf{\large{Éder Rítis Aragão Costa}}\\
            \end{tabular}
    \end{flushright}
    
    \vfill
    \Large{Agosto de 2024}\\[0.2cm]
    \Large{São Carlos}
\end{titlepage}

\newpage

\tableofcontents

\newpage

\section{Teoria da Medida e Espaços \texorpdfstring{$L^p$}{Lp}}

O primeiro passo para o estudo das desigualdades de Clarkson é a construção de alguns conceitos de teoria da medida.

\subsection{O problema da medida em \texorpdfstring{$\mathbb{R}^n$}{Rn}}

Considere a família $\mathcal{P}(\mathbb{R}^n)$ de todos os subconjuntos de $\mathbb{R}^n$. Se $\{a_i\}_{i = 1}^n$ e $\{b_i\}_{i = 1}^n$ são conjuntos em $\mathbb{R}$ tais que $a_i < b_i$ para todo $i = 1, \dots, n$, então podemos definir $$\ell\left(\prod_{i = 1}^n [a_i, b_i]\right) = \prod_{i = 1}^n |b_i - a_i|$$ como o \textit{volume} do $n$-retângulo dado pelo cartesiano acima. O que procuramos a seguir é uma função $\ell \colon \mathcal{P}(\mathbb{R}^n) \to [0, \infty]$ que estenda de maneira ``minimamente razoável'' o volume de $n$-retângulos em $\mathbb{R}^n$. Com ``minimamente razoável'', queremos dizer que a nossa esperança é de que: \begin{itemize}
    \item se $x \in \mathbb{R}^n$ e $A \subset \mathbb{R}^n$, então $$\ell(A + x) = \ell(A);$$
    \item se $\{A_i\}_{i = 1}^n \subset \mathcal{P}(\mathbb{R}^n)$ é tal que $A_i \cap A_j = \emptyset$ para $i \neq j$, então $$\ell\left(\bigcup_{i = 1}^\infty A_i\right) = \sum_{i = 1}^\infty \ell(A_i).$$ 
\end{itemize}
O texto produzido aqui pode ser de cunho acadêmico, mas o que seria de nós, autores, se não ensinássemos alguma coisa sobre a vida fora da matemática? Nossa primeira lição, portanto, é que esperança não serve de nada. Provaremos isso mostrando que tal extensão $\ell$ não pode existir. Para cumprir tal tarefa, definimos uma relação: se $x, y \in \mathbb{R}^n$ escrevemos $$x \sim y \quad \text{se, e somente se,} \quad x - y \in \mathbb{Q}^n.$$

\begin{proposition}
    A relação definida acima é de equivalência.
\end{proposition}
\begin{proof}
    De fato, \begin{itemize}
        \item a relação é simétrica, já que $x - x = 0 \in \mathbb{Q}^n$;
        \item a relação é reflexiva, já que se $x - y \in \mathbb{Q}^n$, então $- (x - y) \in \mathbb{Q}^n$, e $-(x-y) = y-x$;
        \item a relação é transitiva, já que se $x - y, y - z \in \mathbb{Q}^n$, então $x - z = (x - y) - (y - z) \in \mathbb{Q}^n$.
    \end{itemize}
\end{proof}

\nocite{*}
\printbibliography
\end{document}
































